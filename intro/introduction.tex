\documentclass[../main.tex]{subfiles}
\begin{document}
\chapter{Introduction}
The neutrino is well known for how difficult it is to measure. It hardly partakes in the fundamental forces that govern the physics of our universe. Without a charge or a colour charge, for a long time, neutrinos were believed to only partake in the weak force. Yet one of the biggest revelations of the last 40 years was the experimental proof that neutrinos have a mass. This was done by the observation of neutrino oscillation. This is because neutrinos from reactors seem to vanish after travelling long distances. It was later understood that they did not vanish but rather converted from one flavour to another. This had to mean that the neutrinos have a mass difference and thus also a non-zero mass. However, these oscillation measurements are only able to measure the mass difference between the different mass eigenstates of the neutrino. To measure the absolute mass scale this was not enough. This gives however a minimum mass that the sum of the neutrinos would need to have. We find \begin{align*}
     \sum m_\nu > 0.06 \mathrm{eV}
\end{align*}\\
To get a handle on the neutrino mass we have to look at the largest particle accelerator that we know, the big bang. The Neutrinos are believed to have decoupled from the thermal plasma a second after the Big Bang when typical energies were of the order of MeV. But even after decoupling they have a distinct signature on the observables of cosmology. From Big Bang nucleosynthesis up to structure formation in recent times the neutrinos have left their fingerprints for us to search and find.\\
The tightest constraints on the Neutrino mass come from cosmic microwave background experiments combining the temperature and lensing anisotropies with the baryonic acoustic oscillations. The \Planck collaboration finds an upper 95\% confidence limit for the neutrino mass of 
\begin{align*}
    \sum m_\nu < 0.12 \mathrm{eV},
\end{align*}
closing in with the oscillation experiments to find the absolute mass scale. Of course, the bounds from cosmology are model dependent but since there is no real evidence against $\Lambda$CDM the constraints from \Planck are to be taken very seriously.\\
The new upcoming data from ESA's \Euclid mission will help us to weigh in on the challenge of measuring the neutrino mass. It is planned to be the largest galaxy catalogue to date with a sky convergence of 15000 square degrees and an estimated one billion galaxies used to build a tomographic three-dimensional map of the sky. To compare with the sky coverage of {\it DES} and {\it KiDS} is 5000 and 1000 square degree respectively. With all it's data, \Euclid is forecast to measure the matter power spectrum up to 1\% level and give valuable insight into the many open questions of cosmology.\\
In this work, we will not only try to gauge how well the \Euclid satellite will be able to measure the mass of the neutrinos but also question how well \Euclid be able to measure new massless particles that are not included in the standard model parametrized through the famous $N_\mathrm{eff}$. On top of that, we will try to forecast constraints on generic dark energy parametrizations to put limits on accelerated expansion models. Finally, we will check how well \Euclid will be able to measure a certain class of models of modified gravity theories. \\
In this work, we will use different statistical methods, build a validation for our forecasting pipeline, and explore the deep and rich physics in modelling the large-scale structure of the universe.
\end{document}