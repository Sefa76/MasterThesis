\documentclass[oneside]{book}
\usepackage[T1]{fontenc}
\usepackage[utf8]{inputenc}
\usepackage{geometry}   
\usepackage{float}
\usepackage[section]{placeins}
\usepackage{amsmath}
\usepackage{amssymb}
\usepackage{amsfonts}
\usepackage[colorlinks=true, allcolors=blue]{hyperref}
\usepackage{mathtools}
\usepackage[switch, mathlines]{lineno}
\usepackage[usenames,dvipsnames]{xcolor} 
\usepackage[normalem]{ulem}
\usepackage[capitalise,nameinlink]{cleveref}
\usepackage{enumitem}
\usepackage{csquotes}
\usepackage{xspace}
\usepackage{multirow}
\usepackage{tabularx}
\usepackage{color, colortbl}
\usepackage[colorinlistoftodos]{todonotes}
\usepackage{graphicx}

% for writing code blocks 
\usepackage{listings}
%\usepackage{color}

\definecolor{orange}{rgb}{1,0.5,0}
\definecolor{darkorange}{rgb}{0.69,0.33,0.13}
\definecolor{fidcol}{rgb}{0.7,0,0}
\definecolor{mkcol}{rgb}{0.5,0,0.5}
\definecolor{mmcol}{rgb}{0.7,0.17,0.31}
\definecolor{dscol}{rgb}{0.6,0.1,0.2}
\definecolor{mccol}{rgb}{0.2,0.4,0.6}
\definecolor{darkgreen}{rgb}{0.05,0.5,0.06}
\definecolor{carnelian}{rgb}{0.7, 0.11, 0.11}
\definecolor{dkgreen}{rgb}{0,0.6,0}
\definecolor{mauve}{rgb}{0.58,0,0.82}
%table colors
\definecolor{gray}{gray}{0.9}
\definecolor{cyan}{rgb}{0.88,1,1}

\newcommand*{\Euclid}{\textit{Euclid}\xspace}
\newcommand*{\Planck}{\textit{Planck}\xspace}
\newcommand*{\rd}{\mathrm{d}}
\newcommand*{\marktodo}{{\color{mmcol} ::TODO::}\xspace}
\newcommand*{\halofit}{\texttt{HALOFIT}\xspace}
\newcommand*{\hmcode}{\texttt{HMCODE}\xspace}


\begin{document}
 \chapter{The Nonlinear Power Spectrum}
With the Euclid mission we will be able to probe deeply the non-linear regime of the matter power spectrum. The non-linear power spectrum can not be calculated using just perturbation theory, so to predict the dynamics we would resort back to N-body simulations. As running such simulations are computationally expensive we have different codes to obtain the nonlinear power spectrum. In this work we use two different codes called \halofit and \hmcode. Both codes agree with each other on the level of ~6\% but at the level of the \Euclid mission's precession this is already too much. We will show later on how the choice of the non-linear code will bias our parameter inference. Since \hmcode is better at matching N-body simulations of cosmologies with DE and massive Neutrinos then \halofit, we will use it for the forecast for the massive neutrinos while we will use \halofit for the forecast of modified gravity as the fitting formula was done with the power spectrum of \halofit as a pseudo.\\
 Both codes are based on the halo model to predict the non-linear power spectrum. In the first section we will go over the basics of the halo model.\\
The codes differ in that \halofit is a direct functional fit of the power spectrum from the halo model. \hmcode on the other hand is a semi analytical model where the individual ingredients of the halo model are fitted. They are then combined in the context of a modified halo model to better match to N-body simulations. In the second section we will then go over the implementation of \hmcode we used and explain were the critical modeling of massive neutrinos is important.\\

\section{The Halo Model}
The halo model describes how the non-linear power spectrum is calculated using a sum of two terms 
\begin{equation}
    P_{mm}(k,z) = P^\mathrm{1h}(k,z) + P^\mathrm{2h}(k,z).
\end{equation}
The first "one halo term", $P^\mathrm{1h}(k,z)$, dominates at smaller scales and is calculated solely from the intrinsic properties of halos. The second term is called the "two halo term", $P^\mathrm{2h}(k,z)$, it dominates at larger scales and describes the correlation of two halos. To get to this point we will firstly start\\
We start by stating that all the derivation is done at one point of time, but it is the time dependence can be added afterwards.
The main assumption of the halo model, is that the total matter field, $\rho$, can be written as a sum over different halos, i.e. all the matter is inside halos. 
The second assumption is that the density profile of halos, $\rho_H$, are only a function of their mass,$M_i$ and their relative position to their center, $\boldsymbol{x}_i$. These assumptions lead to 
\begin{equation}
\label{eq:halo_denisity_field}
\rho(\boldsymbol{x}) = \sum_i \rho(\boldsymbol{x}-\boldsymbol{x}_i,M_i)=\int \sum_i \delta(M'-M_i)\,\delta^{(3)}(\boldsymbol{x}'-\boldsymbol{x}_i)\,\rho_H(\boldsymbol{x}-\boldsymbol{x}',M')\, \rd^3 \boldsymbol{x}' \, \rd M'
\end{equation}

In the last step we have artificially added a sum over Dirac-delta distributions to pull out the dependence on the particular realization of the universe out of the halo density profile. The sum in the integral itself can be understood as a random realization of an underlying probability density, $\langle \rd n/\rd M \rangle$, i.e., the number density per mass interval $M,M+\rd M$. This can be understood like in the following. Imagine that the universe is separated into small volume bins, $\delta V_i$, and mass bins, $\delta M_j$, such that in each bin there is only the center of one halo with its mass in that particular mass bin. This would define a random variable $S^{ij}$ such that 

\begin{equation*}
S^{ij} = \left\{ \begin{array}{ll}
    1 & \text{when a halo is in the corresponding bins } ij \\
    0 & \, \textrm{otherwise} \\
    \end{array}  \right.    
\end{equation*}
The expectation value of this variable is given by the probability to find a galaxy in this bin. One could either integrate the underlying probability or take the ensemble average of particular realizations. This leads to

\begin{align}
    &&\left\langle S^{ij} \right\rangle &= \int_{\delta M_j} \int_{\delta V_i}  \left\langle \frac{\rd n}{\rd M} \right \rangle(M) \, \rd^3 \boldsymbol{x} \, \rd M \\
    && &\overset{!}{=}  \left\langle \int_{\delta M_j} \int_{\delta V_i} \sum_k \delta(M-M_k)\,\delta^{(3)}(\boldsymbol{x}-\boldsymbol{x}_k)   \, \rd^3 \boldsymbol{x} \, \rd M \right\rangle  \\
    \Longleftrightarrow && \left\langle \frac{\rd n}{\rd M} \right \rangle(M) &= \left\langle \sum_i \delta(M-M_i)\,\delta^{(3)}(\boldsymbol{x}-\boldsymbol{x}_i)  \right \rangle.
\end{align}
This can be used to find a normalization condition for the probability density by evaluating the expectation value of the density field 

\begin{align}
    1 &\overset{!}{=} \frac{1}{\langle \rho \rangle} \left\langle \int \sum_i \delta(M'-M_i)\,\delta^{(3)}(\boldsymbol{x}'-\boldsymbol{x}_i)\,\rho_H(\boldsymbol{x}-\boldsymbol{x}',M')\, \rd^3 \boldsymbol{x}' \, \rd M' \right \rangle \\
    &= \frac{1}{\langle \rho \rangle} \int \left\langle \sum_i \delta(M'-M_i)\,\delta^{(3)}(\boldsymbol{x}'-\boldsymbol{x}_i) \right\rangle\,\rho_H(\boldsymbol{x}-\boldsymbol{x}',M')\, \rd^3 \boldsymbol{x}' \,\rd M' \nonumber\\
    &= \frac{1}{\langle \rho \rangle} \int \left\langle  \frac{\rd n}{\rd M} \right\rangle(M')\,\rho_H(\boldsymbol{x}-\boldsymbol{x}',M')\, \rd^3 \boldsymbol{x}' \,\rd M'  \nonumber\\
    &=  \int \frac{M'}{\langle \rho \rangle} \,\left\langle  \frac{\rd n}{\rd M} \right\rangle(M') \,\rd M' .\label{eq:halo_mass_function_normalization}
\end{align} 
In the first line we have used that the halo profile of a single halo with mass $M'$ and location $\boldsymbol{x}'$ is independent of a particular realization of the universe and such can be factored out of the ensemble average. In the next step we can use this to find an expression for the density contrast,
\begin{align}
    \label{eq:density-contrast}
    \delta(\boldsymbol{x}) &= \frac{\rho(\boldsymbol{x})-\langle \rho \rangle }{\langle \rho \rangle} = \frac{1}{\langle \rho \rangle} \int \left[  \sum_i \delta(M'-M_i)\,\delta^{(3)}(\boldsymbol{x}'-\boldsymbol{x}_i)-\left\langle  \frac{\rd n}{\rd M} \right\rangle(M')\right]\,\rho_H(\boldsymbol{x}-\boldsymbol{x}',M')\, \rd^3 \boldsymbol{x}' \,\rd M'\\
&\coloneqq  \int \left\langle  \frac{\rd n}{\rd M} \right\rangle(M')\,\delta_H(\boldsymbol{x}',M') \,\frac{\rho_H(\boldsymbol{x}-\boldsymbol{x}',M')}{\langle \rho \rangle}\, \rd^3 \boldsymbol{x}' \,\rd M',
\end{align}
Where we have defined a new halo distribution contrast, $\delta_H$, that compares the realization of halo masses and centers to their underlying uniform distribution. The next step is to calculate the two point correlation function, $\xi(\boldsymbol{x}_1,\boldsymbol{x}_2) = \langle \delta(\boldsymbol{x}_1) \delta(\boldsymbol{x}_2) \rangle$. For this we will need to evaluate the same term, but with the halo distribution contrast. We write it as

\begin{align}
    \label{eq:two-point-halo-seed-denstiy-correlation}
    \langle \delta_H(\boldsymbol{x}_1,M_1) \delta_H(\boldsymbol{x}_2,M_2) \rangle =& \left[\left\langle  \frac{\rd n}{\rd M} \right\rangle_1\left\langle  \frac{\rd n}{\rd M} \right\rangle_2\right]^{-1}\,\left\langle\left(\sum_i \delta^{(4)}(\tilde{\boldsymbol{x}}_1-\tilde{\boldsymbol{x}}_i) - \left\langle  \frac{\rd n}{\rd M} \right\rangle_1\right) \right. \\
\times& \left.\left(\sum_j \delta^{(4)}(\tilde{\boldsymbol{x}}_2-\tilde{\boldsymbol{x}}_i) - \left\langle  \frac{\rd n}{\rd M} \right\rangle_2\right) \right\rangle \nonumber\\
=& \left[\left\langle  \frac{\rd n}{\rd M} \right\rangle_1\left\langle  \frac{\rd n}{\rd M} \right\rangle_2\right]^{-1}\,\left[\left\langle\sum_i \delta^{(4)}(\tilde{\boldsymbol{x}}_1-\tilde{\boldsymbol{x}}_i)\sum_j \delta^{(4)}(\tilde{\boldsymbol{x}}_2-\tilde{\boldsymbol{x}}_j)\right\rangle \right. \\
+&\left. \left\langle  \frac{\rd n}{\rd M} \right\rangle_1\left\langle  \frac{\rd n}{\rd M} \right\rangle_2\right]. \nonumber
\end{align}
We have used the shorthand notations $\delta^{(4)}(\tilde{\boldsymbol{x}}-\tilde{\boldsymbol{x}}_i) = \delta^{(3)}({\boldsymbol{x}}-{\boldsymbol{x}}_i)\,\delta(M-M_i)$ for the four dimensional Dirac delta and $\langle {\rd n}/{\rd M} \rangle_i = \langle  {\rd n}/{\rd M}\rangle(M_i)$ for the halo mass function.\\
In the next step we will separate the integrals that show up in the calculation of the two point correlation function into bins like before, this allows us to find the random variables $S^{ij}$ again. For notation's sake we will slice up the integration space into four dimensional volumes with one index. We will then find integrals like

\begin{align}
    \mathcal{I}_g &= \iint g(\tilde{\boldsymbol{x}}_1,\tilde{\boldsymbol{x}}_2)\,\left\langle\sum_i \delta^{(4)}(\tilde{\boldsymbol{x}}_1-\tilde{\boldsymbol{x}}_i)\sum_j \delta^{(4)}(\tilde{\boldsymbol{x}}_2-\tilde{\boldsymbol{x}}_j)\right\rangle\, \rd^4\tilde{\boldsymbol{x}}_1 \, \rd^4\tilde{\boldsymbol{x}}_2 \label{eq:start_derivation_dirac}\\
                &= \sum_\mu \sum_\nu g(\tilde{\boldsymbol{x}}_\mu,\tilde{\boldsymbol{x}}_\nu) \left\langle S^\mu S^\nu \right\rangle,
\end{align} 
with an arbitrary function $g$. In the easy case the indices $\mu$ and $\nu$ are the same, and we find 

\begin{align}
    \langle S^\mu  S^\mu \rangle = \langle S^\mu \rangle &= \int_{\delta V^{(4)}_\mu} \left\langle \frac{\rd n}{\rd M} \right\rangle_1 \,\rd^4 \tilde{\boldsymbol{x}}_1.
\end{align}
For the term with two different indices we need to use the two point correlation function of halo seed densities of equation \ref{eq:two-point-halo-seed-denstiy-correlation}. This leads to 

\begin{align}
    \langle S^\mu  S^\nu \rangle &= \int_{\delta V^{(4)}_\mu} \int_{\delta V^{(4)}_\nu} \langle \delta_H(\tilde{\boldsymbol{x}}_1) \delta_H(\tilde{\boldsymbol{x}}_2) -1 \rangle\,\left\langle \frac{\rd n}{\rd M} \right\rangle_1 \left\langle \frac{\rd n}{\rd M} \right\rangle_2\,\rd^4 \tilde{\boldsymbol{x}}_1 \, \rd^4 \tilde{\boldsymbol{x}}_2\\
&\coloneqq \int_{\delta V^{(4)}_\mu} \int_{\delta V^{(4)}_\nu} ( \xi_H(\tilde{\boldsymbol{x}}_1,\tilde{\boldsymbol{x}}_2) -1 )\,\left\langle \frac{\rd n}{\rd M} \right\rangle_1 \left\langle \frac{\rd n}{\rd M} \right\rangle_2\,\rd^4 \tilde{\boldsymbol{x}}_1 \, \rd^4 \tilde{\boldsymbol{x}}_2.
\end{align}
With this we can find the value of $\mathcal{I}_g$ and relate it to the original integral by joining the bins together. We find
\begin{align}
    \mathcal{I}_g &= \sum_{\mu\,\nu} g(\tilde{\boldsymbol{x}}_\mu,\tilde{\boldsymbol{x}}_\nu)\,\left\langle S^\mu S^\nu \right\rangle\\
     &= \sum_{\mu \neq \nu} \int_{\delta^{(4)}_\mu}\int_{\delta^{(4)}_\nu}g(\tilde{\boldsymbol{x}}_\mu,\tilde{\boldsymbol{x}}_\nu)\,( \xi_H(\tilde{\boldsymbol{x}}_1,\tilde{\boldsymbol{x}}_2) -1 )\,\left\langle \frac{\rd n}{\rd M} \right\rangle_1 \left\langle \frac{\rd n}{\rd M} \right\rangle_2 \,\rd^4 \tilde{\boldsymbol{x}}_1\,\rd^4 \tilde{\boldsymbol{x}}_2\nonumber\\
     &+ \sum_{\mu}  \int_{\delta^{(4)}_\mu} g(\tilde{\boldsymbol{x}}_\mu,\tilde{\boldsymbol{x}}_\mu)\,\left\langle \frac{\rd n}{\rd M} \right\rangle_1\,\rd^4 \tilde{\boldsymbol{x}}_1\\
     &= \sum_{\mu \neq \nu} \int_{\delta^{(4)}_\mu}\int_{\delta^{(4)}_\nu}g(\tilde{\boldsymbol{x}}_1,\tilde{\boldsymbol{x}}_2)\,( \xi_H(\tilde{\boldsymbol{x}}_1,\tilde{\boldsymbol{x}}_2) -1 )\,\left\langle \frac{\rd n}{\rd M} \right\rangle_1 \left\langle \frac{\rd n}{\rd M} \right\rangle_2 \,\rd^4 \tilde{\boldsymbol{x}}_1\,\rd^4 \tilde{\boldsymbol{x}}_2\nonumber\\
     &+ \sum_{\mu}  \int_{\delta^{(4)}_\mu} g(\tilde{\boldsymbol{x}}_1,\tilde{\boldsymbol{x}}_1)\,\left\langle \frac{\rd n}{\rd M} \right\rangle_1\,\rd^4 \tilde{\boldsymbol{x}}_1\\
     &= \iint g(\tilde{\boldsymbol{x}}_1,\tilde{\boldsymbol{x}}_2)\,( \xi_H(\tilde{\boldsymbol{x}}_1,\tilde{\boldsymbol{x}}_2) -1 )\,\left\langle \frac{\rd n}{\rd M} \right\rangle_1 \left\langle \frac{\rd n}{\rd M} \right\rangle_2 \,\rd^4 \tilde{\boldsymbol{x}}_1\,\rd^4 \tilde{\boldsymbol{x}}_2\nonumber\\
     &+ \iint g(\tilde{\boldsymbol{x}}_1,\tilde{\boldsymbol{x}}_2)\,\left\langle \frac{\rd n}{\rd M} \right\rangle_1\,\delta^{(4)}(\tilde{\boldsymbol{x}}_1-\tilde{\boldsymbol{x}}_2)\,\rd^4 \tilde{\boldsymbol{x}}_1\,\rd^4 \tilde{\boldsymbol{x}}_2\\
     &= \iint g(\tilde{\boldsymbol{x}}_1,\tilde{\boldsymbol{x}}_2)\,\left\langle \frac{\rd n}{\rd M} \right\rangle_1\,\left[\left\langle \frac{\rd n}{\rd M} \right\rangle_2\,( \xi_H(\tilde{\boldsymbol{x}}_1,\tilde{\boldsymbol{x}}_2) -1 )+\delta^{(4)}(\tilde{\boldsymbol{x}}_1-\tilde{\boldsymbol{x}}_2)\right]\,\rd^4\tilde{\boldsymbol{x}}_1\,\rd^4\tilde{\boldsymbol{x}}_2 \label{eq:end_detivation_dirac}
\end{align}
In the fourth line we use that the function $g$ is varying slowly in one bin and thus can be approximated as the value at the bin center. We then combine the microcells in line six and artificially add an integration over $\tilde{\boldsymbol{x}}_2$ to find the final result. When comparing equations \ref{eq:start_derivation_dirac} and \ref{eq:end_detivation_dirac} we find an expression for the halo seed density two point correlation function, 
\begin{align}
    \left\langle\sum_i \delta^{(4)}(\tilde{\boldsymbol{x}}_1-\tilde{\boldsymbol{x}}_i)\sum_j \delta^{(4)}(\tilde{\boldsymbol{x}}_2-\tilde{\boldsymbol{x}}_j)\right\rangle = \left\langle \frac{\rd n}{\rd M} \right\rangle_1\,\left[\left\langle \frac{\rd n}{\rd M} \right\rangle_2\,( \xi_H(\tilde{\boldsymbol{x}}_1,\tilde{\boldsymbol{x}}_2) -1 )+\delta^{(4)}(\tilde{\boldsymbol{x}}_1-\tilde{\boldsymbol{x}}_2)\right] \\
    \langle \delta_H(\boldsymbol{x}_1,M_1) \delta_H(\boldsymbol{x}_2,M_2) \rangle = \xi_H(\boldsymbol{x}_1,\boldsymbol{x}_2,M_1,M_2)+ \left[\left\langle \frac{\rd n}{\rd M} \right\rangle\!(M_1) \right]^{-1}\!\! \delta^{(3)}(\boldsymbol{x}_1-\boldsymbol{x}_2)\,\delta(M_1-M_2). \label{eq:two-point-halo-seed-density}
\end{align}
With this we can find the two point correlation function to be
\begin{align}
    \xi(\boldsymbol{x}_1,\boldsymbol{x}_2) =& \iint \left\langle \frac{\rd n}{\rd M} \right\rangle_1\!\!\left\langle  \frac{\rd n}{\rd M} \right\rangle_2\!\!
    \langle\delta_H(\tilde{\boldsymbol{x}}'_1)\,\delta_H(\tilde{\boldsymbol{x}}'_2)\rangle \,\frac{\rho_H(\boldsymbol{x}_1-\boldsymbol{x}'_1,M_1')}{\langle \rho \rangle} \,\frac{\rho_H(\boldsymbol{x}_2-\boldsymbol{x}'_2,M_2')}{\langle \rho \rangle}\, \rd^4\tilde{\boldsymbol{x}}'_1 \rd^4\tilde{\boldsymbol{x}}'_2 \nonumber\\
    =&\iint \xi_H(\boldsymbol{x}'_1,\boldsymbol{x}'_2,M_1',M_2') \left\langle \frac{\rd n}{\rd M} \right\rangle_1\!\!\left\langle \frac{\rd n}{\rd M} \right\rangle_2 \!\!\frac{\rho_H(\boldsymbol{x}_1-\boldsymbol{x}'_1,M_1')}{\langle \rho \rangle} \,\frac{\rho_H(\boldsymbol{x}_2-\boldsymbol{x}'_2,M_2')}{\langle \rho \rangle}\, \rd^4\tilde{\boldsymbol{x}}'_1 \rd^4\tilde{\boldsymbol{x}}'_2 \nonumber \\
    +&\int \left\langle \frac{\rd n}{\rd M} \right\rangle_1\,\frac{\rho_H(\boldsymbol{x}_1-\boldsymbol{x}'_1,M_1')}{\langle \rho \rangle}\,\frac{\rho_H(\boldsymbol{x}_2-\boldsymbol{x}'_1,M_1')}{\langle \rho \rangle} \rd^4 \tilde{\boldsymbol{x}}_1'.
\end{align}
In the next step we note that the halo over densities, $\delta_H$, are biased tracers of the matter over densities, $\delta_m$. We assume that the bias is a function of the mass of the halo but not its location due to the isotropy of space, and is linear. With this we can find \begin{align*}
    \xi_H(\boldsymbol{x}_1,\boldsymbol{x}_2,M_1,M_2) = b(M_1)\,b(M_2)\,\xi_m(\boldsymbol{x}_1,\boldsymbol{x}_2). 
\end{align*}
Our initial assumptions that all the matter should be in halos argues that the bias function should be a function close to unity and thus the integral 
\begin{equation}
\int b(M)\,\left\langle\frac{\rd n}{\rd M}\right\rangle \frac{\rho_H(\boldsymbol{x},M)}{\langle\rho\rangle} \rd^3\boldsymbol{x}\,\rd M \approx 1. \label{eq:two-halo-integral}
\end{equation}
To find the power spectrum we start by the Fourier transformation of the density contrast 

\begin{align}
    \delta(\boldsymbol{k}) = \frac{1}{(2\pi)^3}&\int \left\langle  \frac{\rd n}{\rd M} \right\rangle(M')\,\delta_H(\boldsymbol{x}',M') \,\frac{\rho_H(\boldsymbol{x}-\boldsymbol{x}',M')}{\langle \rho \rangle}\,\exp[{-i\boldsymbol{k}\cdot\boldsymbol{x}}] \rd^3 \boldsymbol{x}' \,\rd M'\,\rd^3\boldsymbol{x}\\
    =&\int \left\langle  \frac{\rd n}{\rd M} \right\rangle(M')\,\delta_H(\boldsymbol{x}',M') \,\frac{\rho_H(\boldsymbol{k},M')}{\langle \rho \rangle}\,\exp[{-i\boldsymbol{k}\cdot\boldsymbol{x}'}] \rd^3 \boldsymbol{x}' \,\rd M'
\end{align} 
In the next step we need to calculate the expectation value of the density contrast. We will then insert the two point correlation of halo density contrast from equation \ref{eq:two-point-halo-seed-density} to find
\begin{align}
    \langle \delta(\boldsymbol{k}_1)\delta^*(\boldsymbol{k}_2)\rangle &=  \int \xi(\boldsymbol{x}_1-\boldsymbol{x}_2) b(M_1)\, b_(M_2)\, \left\langle \frac{\rd n}{\rd M} \right\rangle_1\!\!\left\langle \frac{\rd n}{\rd M} \right\rangle_2 \!\!\frac{\rho_H(\boldsymbol{k}_1,M_1)}{\langle \rho \rangle}\,e^{-i\boldsymbol{k}_1\cdot\boldsymbol{x}_1} \nonumber \\
    &\times \frac{\rho_H(\boldsymbol{k}_2,M_2)}{\langle \rho \rangle}\,e^{i\boldsymbol{k}_2\cdot\boldsymbol{x}_2} \, \rd^3{\boldsymbol{x}}_1 \rd^3{\boldsymbol{x}}_2 \,\rd M_1\,\rd M_2 \nonumber\\
    &+\int \left\langle \frac{\rd n}{\rd M} \right\rangle_1\,\frac{\rho_H(\boldsymbol{k}_1,M_1)}{\langle \rho \rangle}\,\frac{\rho_H(\boldsymbol{k}_2,M_1)}{\langle \rho \rangle}\,e^{-i\boldsymbol{x}_1\cdot[\boldsymbol{k}_1-\boldsymbol{k}_2]} \,\rd^3 \boldsymbol{x}_1 \,\rd M_1 \\
    &=  \int P_{mm}(\boldsymbol{k}') \, b(M_1)\, \left\langle \frac{\rd n}{\rd M} \right\rangle_1 \frac{\rho_H(\boldsymbol{k}_1,M_1)}{\langle \rho \rangle}\,e^{-i \boldsymbol{x}_1\cdot[\boldsymbol{k}_1-\boldsymbol{k}']} \nonumber \\
    &\times  b(M_2)\,\left\langle\frac{\rd n}{\rd M} \right\rangle_2\,\frac{\rho_H(\boldsymbol{k}_2,M_2)}{\langle \rho \rangle}\,e^{i\boldsymbol{x}_2\cdot[\boldsymbol{k}_2-\boldsymbol{k}']} \, \rd^3{\boldsymbol{x}}_1 \rd^3{\boldsymbol{x}}_2 \,\rd M_1\,\rd M_2\,\rd^3\boldsymbol{k}' \nonumber\\
    &+(2\pi)^3\,\int \left\langle \frac{\rd n}{\rd M} \right\rangle_1\,\left[\frac{\rho_H(\boldsymbol{k}_1,M_1)}{\langle \rho \rangle}\right]^2\,\delta(\boldsymbol{k}_1-\boldsymbol{k}_2) \,\rd M_1 \\
    &= (2\pi)^6 \int P_{mm}(\boldsymbol{k}_1) \, b(M_1)\, \left\langle \frac{\rd n}{\rd M} \right\rangle_1 \frac{\rho_H(\boldsymbol{k}_1,M_1)}{\langle \rho \rangle}\,\delta(\boldsymbol{k}_1-\boldsymbol{k}_2) \nonumber \\
    &\times  b(M_2)\,\left\langle\frac{\rd n}{\rd M} \right\rangle_2\,\frac{\rho_H(\boldsymbol{k}_2,M_2)}{\langle \rho \rangle} \,\rd M_1\,\rd M_2\nonumber\\
    &+(2\pi)^3\,\int \left\langle \frac{\rd n}{\rd M} \right\rangle_1\,\left[\frac{\rho_H(\boldsymbol{k}_1,M_1)}{\langle \rho \rangle}\right]^2\,\delta(\boldsymbol{k}_1-\boldsymbol{k}_2)\,\rd M_1.
\end{align}
In the first step we have inserted that the two point correlation function of matter is the Fourier back transformed matter power density. Then we have factored out plane waves such that after integration over real space we are left with Dirac delta distributions, we then use them to simplify the integrals in the last step. After We factor out the remaining Dirac deltas we find the expression of the power spectrum as the sum of the one halo and two halo terms. The first Term is the two halo term and is given by 
\begin{equation}
P^{2h}(k) \coloneqq P_{mm}(k) \left[(2\pi)^3\int b(M)\,\left\langle\frac{\rd n}{\rd M}\right\rangle\!\!(M)\,\frac{\rho(\boldsymbol{k},M)}{\langle \rho \rangle} \, \rd M\right]^2.
\end{equation}
Its interpretation is that it is the correlation of two separate halos. The integral itself can be understood as a mean over the galaxy shapes with a small bias stemming from the fact, that very massive halos cluster much more strongly than normal matter. It is also proportional to the matter power spectrum because these halos are formed from seeds of initial mater perturbations. We infer from this that their spacial correlation is given by the mater power spectrum.\\
The second term is the one halo term. It is given by 
\begin{equation}
    P^{1h}(k) \coloneqq (2\pi)^3 \int \left\langle\frac{\rd n}{\rd M}\right\rangle\!\!(M)\,\left[\frac{\rho(\boldsymbol{k},M)}{\langle \rho \rangle}\right]^2 \,\rd M.
\end{equation}
In the context of galaxy clustering the interpretation of this term would be as shot noise. In this context it is a bit more subtle as it now correlates the matter in one halo with itself. This gives it its functional dependence of $\rho^2$. The integral itself can again be understood as a mean over the halo distribution such that this term becomes the mean self correlation. 

\section{The \hmcode Implementation}
Both components need firstly the Fourier transformation of the halo density profile
\begin{equation}
    \rho_H(k,z,M) = \int e^{i\boldsymbol{k}\cdot \boldsymbol{r}}\,\rho_H(r,z,M)\,\rd^3 \boldsymbol{r} = \int_0^{r_{lim}} 4\,\pi\,r^2 \frac{\sin(k\,r)}{k\,r}\,\rho(r,z,M) \rd r.
\end{equation}
The cutoff radius, $r_\mathrm{lim}$, is defined via the mass of the Halo. It is calculated by requiring that the mass enclosed by a spehre of that radius is given by 

\begin{equation}
    M = \frac{4\,\pi}{3}\,r_\mathrm{lim}^3\, \Delta_H(z)\,\bar{\rho},
\end{equation}
where $\Delta_H$ is a density contrast, describing how much higher the halo density is then the underlying matter density of the universe $\bar{\rho}$. It was calculated in Einstein-de Sitter cosmologies that the Halo after virialization has am overdensity of $\approx$ 178. To account for cosmologies with $\Omega_m$ different from one and massive neutrinos we use a fitting formula from Mead\marktodo. It is fitted to use only parameters that govern the formation of structure like the (integrated) growthrate and $\Omega_m$. To handle massive neutrinos an additional correction factor as a function of $f_\nu$ is added to the fitting formula. To keep the fitting parameters for cosmologies with massive neutrinos and dark energy the same, the growthrates and mass fraction that enter the fitting formula are calculated in an equivalent cosmology where the massive neutrinos and the dark energy are converted to CDM and $\Lambda$ respectively.\\
The halo density profile $\rho$ is taken as a modified FNW \marktodo profile 

\begin{equation}
    \rho(r,z,M) \propto \frac{1}{\frac{r}{r_s\,\nu^\eta}\left(1+\frac{r}{r_s\,\nu^\eta}\right)},
\end{equation}
where the parameter $\nu^\eta$ is a parameter further bloating the halo shape was added. The proportionality factor in front is used to fix the mass of the halo. The parameter $\nu$ is the peak-height variable defined via 

\begin{equation}
\nu = \frac{\delta_c(z)}{\sigma^{cb}_M(z)}.
\end{equation}
The mass $M$ defines a radius $R_\sigma$ over which we calculate the variance of the smoothed field. The relation is setting the scale to the radius a sphere of background matter density with mass $M$. This leads to a definition simmilar to $r_\mathrm{lim}$ \begin{equation*}
    M =  \frac{4\,\pi}{3}\,R_\sigma^3\,\bar{\rho}.
\end{equation*}
The variable $\delta_c$ is the critical overdensity that matter distributions must reach to collapse into a halo. It was computed in Einstein-de Sitter to be close to 1.686. To more accurately model spherical collapse $\delta_c$ is taken as a fitting function from mead\marktodo, and modified to account for the presence of massive neutrinos though a correction factor. Similar to $\Delta_H$ the fitting formula only is a function of the (integrated) growthrate and $\Omega_m$ calculated in an equivalent cosmology in the presence of massive neutrinos and dark energy.\\
The parameter $\nu$ essentially quantifies how likely it is for given matter over densities to collapse. That is why we use the variance of the CDM+baryon field to calculate it, as we know that neutrinos are not slow enough to cluster on scales of halos. It is often encountered in the context of peak-background split formalism and spherical collapse models like the ones from Press-Schächter \marktodo. The parameter $\eta$ is a free parameter in the \hmcode model and is fitted to be $\eta(z)=\mathcal{A}_\eta\times\left[\sigma_8^{cb}(z)\right]^{\alpha_\eta}$. The functional form is chosen like this because it should be a function of a parameter that actually describes the collapse of halo and not just cosmological parameters.\\
Finally, The halo shape radius, $r_s$, parametrizes the radial profile of the halo density, and it is related to the boundary of the halo, $r_\mathrm{lim}$, via the halo concentration, $c$, \begin{align*}
    r_\mathrm{lim} = c\,r_s.
\end{align*}
This parameter effects the innermost structure of the halo and is thus effecting the matter power spectrum on smallest scales. We take its functional form to be

\begin{equation}
    c(M,z) = \mathcal{B}\left[\frac{1+z_\mathrm{f}(M,z)}{1+z}\right]\,\frac{D(z_c)}{D^\mathrm{eqq}(z_c)}\,\frac{D(z)}{D^\mathrm{eqq}(z)}
\end{equation}
The factor $\mathcal{B}$ is a free constant that was fitted to the N-body simulations. The term $z_\mathrm{f}$ that appears in the equation is the redshift at which the halo is formed. We consider a halo of mass $M$ to be formed when its innermost part crosses the critical density. If we then assume normal growth we find the implicit equation

\begin{equation}
    \frac{D(z_\mathrm{f})}{D(z)}\,\sigma^{cb}_{\gamma M }(z) = \delta_c(z),
\end{equation}
with the free parameter $\gamma$ chosen to be $1\%$, defining what innermost part actually means. The formula is equivalent to asking when a smaller halo of mass $\gamma\,M$ would have formed. The redshift of formation $z_\mathrm{f}$ often ends up being very high, indicating, that the properties of halos are set very early in their evolution. Still in some cases $z_\mathrm{f}$ could end up being earlier than the redshift $z$. In these cases we set $z_\mathrm{f}=z$. The next two factors account for the effect of dark energy further modifying the dynamics of collapse. The first is an empirical correction factor that was obtained from fits of the NFW profile to N-body simulations by Dolag et al \marktodo. The redshift of collapse $z_c$ would be calculated similarly as $z_f$, but we can use that the ratio becomes constant at high redshifts and just use a high redshift of $z_c=10$. For high redshifts $z$ this factor would need to vanish, as then the effect of dark energy is negligible. The second fraction was added for this purpose in \hmcode.\\
The next ingredient to calculate the one-halo term and the two-halo term is the halo mass function, $\frac{\rd n}{\rd M}$. As we know from equation \ref{eq:halo_mass_function_normalization} it has to be normalized.  That is why it is more convenient to use the normalized halo mass function $F(\nu,z)$. We take it as

\begin{equation}
    F(\nu,z) \,\rd \nu= \frac{M}{\bar{\rho}}\, \frac{\rd n}{\rd M} \, \rd M = A\,\left[1+\frac{1}{\left(q\,\nu^2\right)^p}\right]\,e^{-q\,\nu^2/2} \,\rd \nu.
\end{equation}
It is a modified version of the Press Schechter function with free parameters $q$ and $p$. The parameter $A$ is calculated from the normalization. We use the standard values for these 

\begin{equation*}
    p=0.3,\quad q=0.707,\quad A=0.21616.
\end{equation*}
The final ingredients that we would need to calculate the two halo term would be the halo mass bias. To calculate it one would need to follow the peak-background split formalism, where we assume that all matter perturbation that cross the critical overdensity collapse into halos. The bias itself asks, how much stronger more massive halos cluster and can thus be calculated from the derivative of the halo mass function with respect to the peak-height variable.
In reality, we do not need the bias to calculate the two halo term. Like stated before the bias is a function close to unity and thus the integral that goes into the two halo term approximates to 1, as seen in equation \ref{eq:two-halo-integral}.\\
To better match simulations the final formula for the two-halo term is modified in two regards. Firstly the power spectrum in the two halo term is replaced by a power spectrum where the wiggles of the BAO has been smoothed out to an extent. This is a well understood effect of gravitational evolution. The exact formulation of the "de-wiggling" is topic of current debate, but a simple approach is followed in \hmcode.\\
 We start with the linear power spectrum obtained by our Boltzmann solver, $P_{mm}^\mathrm{lin}$, We then divide it with the Eisenstein-Hu no wiggle power spectrum approximation. We then smooth the ratio by coevolving it with a broad Gaussian filter. After multiplying again with the approximation we find the smoothed power spectrum, $P_\mathrm{smt}$. The "de-wiggled" power spectrum is given then as a weighted sum\begin{align}
    P_{dw} &= e^{-g(k,z)}\,P^{lin}_{mm}(k,z) + (1-e^{-g(k,z)})\,P_\mathrm{smt}(k,z)\\
 g(k,z) &= k^2\,\frac{1}{6\,\pi^2}\int_0^\infty P^\mathrm{lin}_{mm}(k,z)\,\rd k\coloneqq k^2 \sigma_v^2(z).
 \end{align}
The second modification of the two halo spectrum, is that by default it does not contain an early dip of the power spectrum that occurs in the one-loop correction in effective field theory approach of the LSS. Its physical interpretation is that the growth of voids in reality is slower than predicted in linear perturbation theory and lead to an overestimate of the power spectrum on very large scales. This effect is modeled by a multiplication of the de-wiggled power spectrum with a function fitted to dampen the power spectrum by a factor $(1-f_D)$ past a scale $k_D$.\\
The final two halo power spectrum in \hmcode is given by \begin{equation}
    P^{2h}(k,z) = P_\mathrm{dw}(k,z)\,\left[1-f_D(z)\,\frac{k^{n_D}(z)}{k_D^{n_D}(z)+k^{n_D}}\right].
\end{equation}
The functions $k_D(z)$ and $f_D(z)$ have the same functional form as the shape parameter, $\eta(z)$, that appears in the FNW profile. We again use the variance of the CDM+baryon field in the functional form to fit as massive neutrinos should not affect the universal collapse dynamics. The parameter $n_D$ is also a free parameter and is fitted to the N-body simulations.\\
The one halo term has also two modification done to it compared to the one obtained from the halo model. Firstly in equation \ref{eq:density-contrast} we assume that halos make up the matter density contrast of the universe. Since only cold matter have cluster in halos they make up only CDM+baryon contrast, we need to correct for this and multiply the halo density profile with a factor $(1-f_\nu)$.\\
The second modification we need to do is a suppression of the one halo term on large scales. It turns out that the one halo term in the standard halo model becomes constant on largest scales, but because of energy conservation a virilized halo should lead to a power spectrum falling of like $k^4$. To account for this we multiply the one halo power spectrum such that our final formula is given by
\begin{equation}
    P^{1h}(k,z) = (1-f_\nu)^2\,(2\,\pi)^3\int_0^\infty \left\langle\frac{\rd n}{\rd M}\right\rangle\!\!(M)\,\left[\frac{\rho(k,M)}{\langle \rho \rangle}\right]^2 \,\rd M\,\frac{k^4}{k_*(z)^4+k^4}.
\end{equation}
The transition wave number, $k_*$, is fitted to a functional form like the transition wave number of the two halo term $k_D$.\\
A final problem of the halo model is, that it is not well suited to describe the correlation in the transition region from the one halo term to the two halo term. This is due to neglecting that very close halos, and even over lapping subhalos correlate with each other. Neglecting these effects leads to an up to 20\% underestimate of the power spectrum. To better model the transition we have added a smoothing parameter $\alpha(z)$. As this parameter effects the spectral index of the power spectrum, its functional form is not a function of the variance of the CDM+baryon field but of the effective spectral index of its power spectrum, $n_\mathrm{eff}$. To calculate it we can use the definition of the variance in real space to find 
\begin{equation}
    3+n_\mathrm{eff}(z) = - \frac{\rd \log \sigma_{cb}(R,z)}{\rd \log R}(R_c,z),
\end{equation}
where $R_c$ is defined as the radius where the standard derivation of the CDM-baryon field crosses the critical overdensity $\delta_c(z)$. The final fitting formula for $\alpha$ is then \begin{equation}
    \alpha(z) = \mathcal{A}_\alpha\,\left[\mathcal{C}_\alpha\right]^{n_\mathrm{eff}(z)}.
\end{equation}
With the transition smoothing we define our nonlinear power spectrum as \begin{equation}
    P^{\mathrm{nl}}_{mm}(k,z) = \left[\left(P^{1h}(k,z)\right)^{\alpha(z)} + \left(P^{2h}(k,z)\right)^{\alpha(z)} \right]^{1/\alpha(z)} .
\end{equation}
\end{document}